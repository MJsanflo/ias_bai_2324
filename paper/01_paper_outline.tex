\documentclass[12pt]{article}
%%%%%%%%%%%%%%%%%%%%%%%%%%%%%%%%%%%%%%%%%%%%%%%%%%%%%%%%%%%%%
% Meta informations:
\newcommand{\trauthor}{Manuel Sandoval Flores}
\newcommand{\trtype}{Proposal} %{Expos\'{e}} %{Review}
\newcommand{\trtitle}{Training A Language Model (GPT) From Scratch And Harnessing It}
\newcommand{\trdate}{09.10.2023}

%%%%%%%%%%%%%%%%%%%%%%%%%%%%%%%%%%%%%%%%%%%%%%%%%%%%%%%%%%%%%
% Languages:

% Falls die Ausarbeitung in Deutsch erfolgt:
% \usepackage[german]{babel}
% \usepackage[T1]{fontenc}
% \usepackage[latin1]{inputenc}
% \usepackage[latin9]{inputenc}	 				
% \selectlanguage{german}

% If the thesis is written in English:
\usepackage[english]{babel} 						
\selectlanguage{english}

%%%%%%%%%%%%%%%%%%%%%%%%%%%%%%%%%%%%%%%%%%%%%%%%%%%%%%%%%%%%%
% Bind packages:
\usepackage{acronym}                    % Acronyms
\usepackage{algorithmic}								% Algorithms and Pseudocode
\usepackage{algorithm}									% Algorithms and Pseudocode
\usepackage{amsfonts}                   % AMS Math Packet (Fonts)
\usepackage{amsmath}                    % AMS Math Packet
\usepackage{amssymb}                    % Additional mathematical symbols
\usepackage{amsthm}
\usepackage{booktabs}                   % Nicer tables
%\usepackage[font=small,labelfont=bf]{caption} % Numbered captions for figures
\usepackage{color}                      % Enables defining of colors via \definecolor
\definecolor{uhhRed}{RGB}{254,0,0}		  % Official Uni Hamburg Red
\definecolor{uhhGrey}{RGB}{122,122,120} % Official Uni Hamburg Grey
\usepackage{fancybox}                   % Gleichungen einrahmen
\usepackage{fancyhdr}										% Packet for nicer headers
%\usepackage{fancyheadings}             % Nicer numbering of headlines

%\usepackage[outer=3.35cm]{geometry} 	  % Type area (size, margins...) !!!Release version
%\usepackage[outer=2.5cm]{geometry} 		% Type area (size, margins...) !!!Print version
%\usepackage{geometry} 									% Type area (size, margins...) !!!Proofread version
\usepackage[outer=3.15cm]{geometry} 	  % Type area (size, margins...) !!!Draft version
\geometry{a4paper,body={5.8in,9in}}

\usepackage{graphicx}                   % Inclusion of graphics
%\usepackage{latexsym}                  % Special symbols
\usepackage{longtable}									% Allow tables over several parges
\usepackage{listings}                   % Nicer source code listings
\usepackage{multicol}										% Content of a table over several columns
\usepackage{multirow}										% Content of a table over several rows
\usepackage{rotating}										% Alows to rotate text and objects
\usepackage[hang]{subfigure}            % Allows to use multiple (partial) figures in a fig
%\usepackage[font=footnotesize,labelfont=rm]{subfig}	% Pictures in a floating environment
\usepackage{tabularx}										% Tables with fixed width but variable rows
\usepackage{url,xspace,boxedminipage}   % Accurate display of URLs

%%%%%%%%%%%%%%%%%%%%%%%%%%%%%%%%%%%%%%%%%%%%%%%%%%%%%%%%%%%%%
% Configurationen:

\hyphenation{whe-ther} 									% Manually use: "\-" in a word: Staats\-ver\-trag

%\lstloadlanguages{C}                   % Set the default language for listings
\DeclareGraphicsExtensions{.pdf,.svg,.jpg,.png,.eps} % first try pdf, then eps, png and jpg
\graphicspath{{./src/}} 								% Path to a folder where all pictures are located
\pagestyle{fancy} 											% Use nicer header and footer

% Redefine the environments for floating objects:
\setcounter{topnumber}{3}
\setcounter{bottomnumber}{2}
\setcounter{totalnumber}{4}
\renewcommand{\topfraction}{0.9} 			  %Standard: 0.7
\renewcommand{\bottomfraction}{0.5}		  %Standard: 0.3
\renewcommand{\textfraction}{0.1}		  	%Standard: 0.2
\renewcommand{\floatpagefraction}{0.8} 	%Standard: 0.5

% Tables with a nicer padding:
\renewcommand{\arraystretch}{1.2}

%%%%%%%%%%%%%%%%%%%%%%%%%%%%
% Additional 'theorem' and 'definition' blocks:
\theoremstyle{plain}
\newtheorem{theorem}{Theorem}[section]
%\newtheorem{theorem}{Satz}[section]		% Wenn in Deutsch geschrieben wird.
\newtheorem{axiom}{Axiom}[section] 	
%\newtheorem{axiom}{Fakt}[chapter]			% Wenn in Deutsch geschrieben wird.
%Usage:%\begin{axiom}[optional description]%Main part%\end{fakt}

\theoremstyle{definition}
\newtheorem{definition}{Definition}[section]

%Additional types of axioms:
\newtheorem{lemma}[axiom]{Lemma}
\newtheorem{observation}[axiom]{Observation}

%Additional types of definitions:
\theoremstyle{remark}
%\newtheorem{remark}[definition]{Bemerkung} % Wenn in Deutsch geschrieben wird.
\newtheorem{remark}[definition]{Remark} 

%%%%%%%%%%%%%%%%%%%%%%%%%%%%
% Provides TODOs within the margin:
\newcommand{\TODO}[1]{\marginpar{\emph{\small{{\bf TODO: } #1}}}}

%%%%%%%%%%%%%%%%%%%%%%%%%%%%
% Abbreviations and mathematical symbols
\newcommand{\modd}{\text{ mod }}
\newcommand{\RS}{\mathbb{R}}
\newcommand{\NS}{\mathbb{N}}
\newcommand{\ZS}{\mathbb{Z}}
\newcommand{\dnormal}{\mathit{N}}
\newcommand{\duniform}{\mathit{U}}

\newcommand{\erdos}{Erd\H{o}s}
\newcommand{\renyi}{-R\'{e}nyi}

%%%%%%%%%%%%%%%%%%%%%%%%%%%%%%%%%%%%%%%%%%%%%%%%%%%%%%%%%%%%%
% Document:
\begin{document}
\renewcommand{\headheight}{14.5pt}

\fancyhead{}
\fancyhead[CO]{\trtitle}

%%%%%%%%%%%%%%%%%%%%%%%%%%%%
% Cover Header:
\title{\trtitle\\[0.3cm]{\normalsize\trtype}}
\author{\trauthor}
\date{\trdate}
\maketitle

%%%%%%%%%%%%%%%%%%%%%%%%%%%%

\thispagestyle{empty}
\pagenumbering{arabic}

% Abstract gives a brief summary of the main points of a paper:
\begin{abstract}
- [1] Provide a concise summary of the paper's key objectives, methods, and findings.
- [2] Include the primary experiment contributions and implications of the research experiment.
\end{abstract}

% the actual content, usually separated over a number of sections
% each section is assigned a label, in order to be able to put a
% crossreference to it

\section{Introduction}
\label{sec:introduction}

- [1] Introduce the seminar project's aim: comparing standard training and curriculum learning for GPT models and analyzing the benefits.
- [2] Highlight the significance of GPT models and the limitations of traditional training approaches.
- [3] Reference the key papers that have inspired this project and analyze approaches in the suggested papers.

\section{Background Information}
\label{sec:back_info}

[-] [Optional] Provide introduction to GPT models and their role in natural language processing.
[1] Discuss the challenges with traditional training methods, which necessitate alternative approaches like curriculum learning.

\section{Related Work}
\label{sec:rel_work}

[1] Summarize the contributions of relevant papers:
- Paper 2 "Curriculum Learning": Eg. Discuss the concept of curriculum learning in deep learning and its potential advantages.
- Paper 3 "Limits of Transformers on Compositionality": Eg. Highlight challenges GPT models face in handling compositional tasks.
- Paper 4 "Locating and Editing Factual Associations in GPT": Eg. Explain how this paper addresses GPT model interpretability.
- Paper 5 "Sparse Autoencoders in Language Models": Eg.Describe the findings related to interpretable features in language models.

\section{Model Description}
\label{sec:model_desc}

[1] Describe the provided minimal GPT model, including its architecture and core components.
[2] Explain the source of training data and any preprocessing steps, such as tokenization or data cleaning.

\section{Model Analysis}
\label{sec:model_analysis}

[1] Explain the standard training approach using randomly shuffled text data.
[2] Describe the curriculum learning strategy used and its implementation for gradually increasing example difficulty.
[3] Detail the experimental setup, specifying hyperparameters, the choice of optimizer, and any unique implementation choices.

\section{Results And Discussion}
\label{sec:result_discussion}

[1] Present a comparison of the GPT model's performance under standard and curriculum learning.
[2] Analyze the benefits of curriculum learning, such as faster convergence, improved generalization, and enhanced performance on complex tasks.
[3] Discuss any challenges faced during the experiments, including data selection and curriculum design.

\section{Understanding GPT Behaviour}
\label{sec:gpt_behaviour}

[1] Summarize key findings from Paper 4 (Locating and Editing Factual Associations in GPT) and Paper 5 (Sparse Autoencoders in Language Models).
[2] Provide insights into the inner workings of GPT models during and after training, particularly focusing on interpretability and feature analysis.

\section{Conclusion}
\label{sec:concl}

[1] Summarize the primary findings and contributions of the experiment project.
[2] Discuss the implications of the results for the field of natural language processing and deep learning referencing the initial motivational questions.
[3] Suggest future research experiment directions, such as exploring advanced curriculum strategies and further improving GPT model interpretability.

%%%%%%%%%%%%%%%%%%%%%%%%%%%%%%%%%%%%%%
% hier werden - zum Ende des Textes - die bibliographischen Referenzen
% eingebunden
%
% Insbesondere stehen die eigentlichen Informationen in der Datei
% ``bib.bib''
%
\section{References}
\label{sec:references}
\begin{thebibliography}{9}

    \bibitem{paper3} Nouha Dziri and Ximing Lu and Melanie Sclar and Xiang Lorraine Li and Liwei Jiang and Bill Yuchen Lin and Peter West and Chandra Bhagavatula and Ronan Le Bras and Jena D. Hwang and Soumya Sanyal and Sean Welleck and Xiang Ren and Allyson Ettinger and Zaid Harchaoui and Yejin Choi. "{Faith and Fate: Limits of Transformers on Compositionality.}" \textit{37th Conference on Neural Information Processing Systems (NeurIPS 2023)}, 2023. arXiv:2305.18654v3

    \bibitem{paper4} Kevin Meng, David Bau, Alex Andonian, Yonatan Belinkov. "Locating and Editing Factual Associations in GPT." \textit{36th Conference on Neural Information Processing Systems (NeurIPS 2022)}, 2022. arXiv:2202.05262v5

    \bibitem{paper5} Hoagy Cunningham, Aidan Ewart, Logan Riggs, Robert Huben, Lee Sharkey. "Sparse Autoencoders Find Highly Interpretable Features in Language Models." \textit{ICML '09: Proceedings of the 26th Annual International Conference on Machine Learning}, 2023. arXiv:2309.08600v3

    \bibitem{paper2} Bengio, Yoshua and Louradour, J\'{e}r\^{o}me and Collobert, Ronan and Weston, Jason. "Curriculum learning." \textit{None}, 2009. https://doi.org/10.1145/1553374.1553380

\end{thebibliography}

\end{document}



